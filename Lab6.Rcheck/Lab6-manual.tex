\nonstopmode{}
\documentclass[a4paper]{book}
\usepackage[times,inconsolata,hyper]{Rd}
\usepackage{makeidx}
\usepackage[utf8,latin1]{inputenc}
% \usepackage{graphicx} % @USE GRAPHICX@
\makeindex{}
\begin{document}
\chapter*{}
\begin{center}
{\textbf{\huge Package `Lab6'}}
\par\bigskip{\large \today}
\end{center}
\begin{description}
\raggedright{}
\item[Type]\AsIs{Package}
\item[Title]\AsIs{knapsack-algos}
\item[Version]\AsIs{1.0}
\item[Date]\AsIs{2015-10-06}
\item[Author]\AsIs{Niclas Lovsj<c3><b6>, Maxime Bonneau}
\item[Maintainer]\AsIs{}\email{niclas.lovsjo@me.com}\AsIs{}
\item[Description]\AsIs{tests different approaches to solving the knapsack-problem. Incl. a brute force, dynamic programming and greedy algorithm}
\item[License]\AsIs{GPL-2}
\item[Suggests]\AsIs{testthat, knitr, rmarkdown}
\item[VignetteBuilder]\AsIs{knitr}
\item[NeedsCompilation]\AsIs{no}
\end{description}
\Rdcontents{\R{} topics documented:}
\inputencoding{utf8}
\HeaderA{Lab6-package}{knapsack-algos}{Lab6.Rdash.package}
\aliasA{Lab6}{Lab6-package}{Lab6}
\keyword{package}{Lab6-package}
%
\begin{Description}\relax
tests different approaches to solving the knapsack-problem. Incl. a brute force, dynamic programming and greedy algorithm
\end{Description}
%
\begin{Details}\relax

The DESCRIPTION file:

\Tabular{ll}{
Package: & Lab6\\{}
Type: & Package\\{}
Title: & knapsack-algos\\{}
Version: & 1.0\\{}
Date: & 2015-10-06\\{}
Author: & Niclas Lovsjö, Maxime Bonneau\\{}
Maintainer: & <niclas.lovsjo@me.com>\\{}
Description: & tests different approaches to solving the knapsack-problem. Incl. a brute force, dynamic programming and greedy algorithm\\{}
License: & GPL-2\\{}
Suggests: & testthat,
knitr,
rmarkdown\\{}
VignetteBuilder: & knitr\\{}
}

Index of help topics:
\begin{alltt}
Brute force             Brute force algorithm for the knapsack problem
Brute force parallel    Brute force algorithm for the knapsack problem
                        using parallelizing
Dynamic programming     Dynamic programming algorithm for the knapsack
                        problem
Greedy heuristic        Greedy heuristic for the knapsack problem
Lab6-package            knapsack-algos
\end{alltt}

\textasciitilde{}\textasciitilde{} An overview of how to use the package, including the most important \textasciitilde{}\textasciitilde{}
\textasciitilde{}\textasciitilde{} functions \textasciitilde{}\textasciitilde{}
\end{Details}
%
\begin{Author}\relax
Niclas Lovsjö, Maxime Bonneau

Maintainer: <niclas.lovsjo@me.com>
\end{Author}
%
\begin{References}\relax
\textasciitilde{}\textasciitilde{} Literature or other references for background information \textasciitilde{}\textasciitilde{}
\end{References}
%
\begin{SeeAlso}\relax
\textasciitilde{}\textasciitilde{} Optional links to other man pages, e.g. \textasciitilde{}\textasciitilde{}
\textasciitilde{}\textasciitilde{} \code{\LinkA{<pkg>}{<pkg>}} \textasciitilde{}\textasciitilde{}
\end{SeeAlso}
%
\begin{Examples}
\begin{ExampleCode}
~~ simple examples of the most important functions ~~
\end{ExampleCode}
\end{Examples}
\inputencoding{utf8}
\HeaderA{Brute force}{Brute force algorithm for the knapsack problem}{Brute force}
%
\begin{Description}\relax
uses brute force, i.e. tests all combinations and finds the one with max value under the restriction total weight<W.
\end{Description}
%
\begin{Arguments}
\begin{ldescription}
\item[\code{x}] is a 2dim matrix containing the weights and values

\item[\code{W}] is the capacity of the knapsack
\end{ldescription}
\end{Arguments}
%
\begin{Value}
a list with the \_value\_ of the optimally packed knapsack and the \_elements\_ that gives this value.
\end{Value}
%
\begin{References}\relax
https://en.wikipedia.org/wiki/Knapsack\_problem
\end{References}
\inputencoding{utf8}
\HeaderA{Brute force parallel}{Brute force algorithm for the knapsack problem using parallelizing}{Brute force parallel}
%
\begin{Description}\relax
uses brute force, i.e. tests all combinations and finds the one with max value under the restriction total weight<W.

uses brute force, i.e. tests all combinations and finds the one with max value under the restriction total weight<W.
\end{Description}
%
\begin{Arguments}
\begin{ldescription}
\item[\code{x}] is a 2dim matrix containing the weights and values

\item[\code{W}] is the capacity of the knapsack

\item[\code{x}] is a 2dim matrix containing the weights and values

\item[\code{W}] is the capacity of the knapsack
\end{ldescription}
\end{Arguments}
%
\begin{Value}
a list with the \_value\_ of the optimally packed knapsack and the \_elements\_ that gives this value.

a list with the \_value\_ of the optimally packed knapsack and the \_elements\_ that gives this value.
\end{Value}
%
\begin{References}\relax
https://en.wikipedia.org/wiki/Knapsack\_problem

https://en.wikipedia.org/wiki/Knapsack\_problem
\end{References}
\inputencoding{utf8}
\HeaderA{Dynamic programming}{Dynamic programming algorithm for the knapsack problem}{Dynamic programming}
%
\begin{Description}\relax
Uses DP to find optimal value and elements, i.e. divides the problem into subproblems and solve each one, memoizes it and solve the whole problem by using that
\end{Description}
%
\begin{Arguments}
\begin{ldescription}
\item[\code{x}] is a 2dim matrix containing the weights and values

\item[\code{W}] is the capacity of the knapsack
\end{ldescription}
\end{Arguments}
%
\begin{Value}
a list with the \_value\_ of the optimally packed knapsack and the \_elements\_ that gives this value.
\end{Value}
%
\begin{References}\relax
https://en.wikipedia.org/wiki/Knapsack\_problem
\end{References}
\inputencoding{utf8}
\HeaderA{Greedy heuristic}{Greedy heuristic for the knapsack problem}{Greedy heuristic}
%
\begin{Description}\relax
Uses greedy heuristic to solve knapsack problem, i.e. orders x by ratio v/w and
picks up the first lines until the knapsack is full (or almost)
\end{Description}
%
\begin{Arguments}
\begin{ldescription}
\item[\code{x}] is a 2dim matrix containing the weights and values

\item[\code{W}] is the capacity of the knapsack
\end{ldescription}
\end{Arguments}
%
\begin{Value}
a list with the \_value\_ of the optimally packed knapsack and the \_elements\_ that gives this value.
\end{Value}
%
\begin{References}\relax
https://en.wikipedia.org/wiki/Knapsack\_problem\#Greedy\_approximation\_algorithm
\end{References}
\printindex{}
\end{document}
